\chapter{Conclusion}
Event-driven architecture and event stream processing is one of the prominent approaches to build up an IT infrastructure which is scalable and robust to changes. By placing the event streams in the center of the entire system to serve as the central data backbone, all data systems as well as applications and services can have a uniform communication channel to exchange data while being loosely coupled. However, there are currently many technologies which enables the construction of systems evolving around the event stream processing platform, each of which has different advantages and disadvantages. There is currently lack of a systematic guideline to evaluate and choose the suitable platform to employ that can match different requirements in different use cases. Therefore, the main aim of the thesis is to derive an evaluation scheme and detail analysis of the available open-source platforms based on this scheme to help assist in the decision-making process when employing and integrating an ESP platform technology into the infrastructure.

In the thesis, three open-source ESP platform technologies which are well-known and widely-used in the community, namely, Apache Kafka, Apache Pulsar, and NATS Streaming, were selected for evaluation. For the assessment, a set of criteria which are arranged into different categories covering a wide range of characteristics from ESP-specific functionalities to general software quality is derived. The evaluation results were summarized in the form of a feature matrix with adjustable weighting factor of each criterion. The matrix can be used to select the most suitable platform among the three technologies based on a predefined set of prioritized requirements. Moreover, through the implementation and in-depth analysis of a simple banking transaction use case and a number of failure scenarios on all three platforms, it can be seen that while Apache Kafka and Apache Pulsar can both provide exactly-once semantics, the guarantee mechanism on Pulsar can lead to out-of-order processing of messages which can be problematic for event-driven use cases. On the other hand, with Apache Kafka, its internal workflow can ensure both the exactly-once semantics and strong ordering guarantee when processing messages. 

For further improvement and adaptation in the future, as these technologies are constantly updated with new features and enhancement in each new release, the derived feature matrix also needs to be regularly refined and adapted with these changes to be kept up-to-date. Moreover, due to the limited scope of the thesis, only three platforms were selected among many available options for ESP platforms and also not all derived assessment criteria were employed for evaluation. Therefore, the matrix can also be expanded to consider more different platforms including both open-source as well as commercial options. More evaluation criteria can also be considered in the evaluation process to give users a wider range of selections and flexibility to choose the right technology for them. Regarding the implementation of the simple use case, in the thesis, only a small number of failure scenarios are considered and analyzed on the platforms. Therefore, this implemented use case can serve further as a reference or a playground to conduct more analyses and tests on the platforms.


 

