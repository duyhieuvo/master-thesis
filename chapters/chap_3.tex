\chapter{Evaluation Scheme} \label{chap:evaluation-scheme}

\section{Considered Platforms}
As the concept of using event streams as the source of truth gains more attention and becomes more popular, many projects aiming at creating a processing platform based on streams started to take shape. 

Numerous companies first started their projects as in-house products and then later open-sourced them to enhance the development pace with the help of community. Kafka, which was first developed at LinkedIn, is the first prominent name in the field. It was later open sourced to the Apache Software Foundation. Yahoo! also created their own stream processing platform named Pulsar and it is now also an Apache project. The company Alibaba joins the trend as well by open sourcing their RocketMQ to Apache Foundation. In addition, there is the NATS streaming server, which is created by Synadia and is currently an incubating project of Cloud Native Computing Foundation. Pravega is also a new name with fast advancement.

Since an adequate evaluation for all these platforms could not be contained within the scope of the thesis, only three platforms will be selected based on a set of criteria indicating the maturity and the size of the active community.

First of all is the 



\section{Evaluation Metrics}

\section{Structure of Feature Matrix}



