\chapter{Introduction} \label{chap:intro}
Nowadays, the explosion of the number of digital devices and online services comes along with an immense amount of data that is auto generated or collected from the interactions of users. For instance, from 2016, the Netflix company already gathered around 1.3 PB of log data on a daily basis \cite{netflixpipeline}. With this unprecedented scale of input data, companies and organizations have tremendous opportunities to utilize them to create business values. Many trending technologies such as Big Data, Internet of Things, Machine Learning and Artificial Intelligent all involves handling data in great volume. However, this also brings about a challenge to collect these data fast and reliably.

%Nowadays there are hypes around the globe about Big Data, Internet of Things, Artificial Intelligent (maybe reference) => what underline all of that is data. Data is the key and we need a huge amount of data. => Where are those data coming from? Machine auto generated, data from people interact with machines and services => The question is how to collect and integrate those huge amount of data (reference amazon has to deal with hug data) into our system and extract value out of it? One of the method to tackle this challenge is represent the flow of data as stream which can be ingest, process along the stream and produce new stream (reference here). 

Once the data is ingested into the organization, it needs to be transformed  and processed to extract insights and generate values. In the context of enterprise applications, as the systems grows over time with more services, the need for an effective data backbone to serve these huge amount of data to these services and to integrate them together while maintaining a good level of decoupling becomes inevitable.

Moreover, all these steps of collecting, processing and transferring data must be done in real-time fashion. One of the prominent methods is \acrfull{esp} which treats data as a continuous flow of events and use this as the \emph{central nervous system} of the software systems with event-driven architecture.

%A system following this concept must receive, persist and process data at a huge scale.  There are many technology to build up stream processing platform. People getting started will be flushed with many technology, platform and framework to persist the event, to process data in real time. 
\section{Motivation}
To develop a system evolving around streams of events, the primary basis is a central event store which can ingest data from multiple sources and serve this data to any interested consumer. Usually an \acrshort{esp} platform will be used as it is designed orienting to the concept of streaming. However, in order to choose the suitable platform, users will usually be burdened by a plethora of questions which need to be answered. The concerns include how well is the performance and reliability of the platform, does the platform provide necessary functionalities, will it deliver messages with accuracy that meets the requirements, can the platform integrate with the existing stream processing framework in the infrastructure, to name but a few.

As there are many platforms now available on the market both open-source and commercial with each having different pros and cons, it could be challenging and time-consuming to go through all of them to choose the most suitable option that matches the requirements. It would be greatly convenient to have a single standardized evaluation of these platforms which can be used as a guideline during the decision-making process. Therefore, the goal of this thesis is to derive a feature matrix to help systematically determine the right open-source \acrshort{esp} platform based on varying priority in different use cases. 

%Moreover, many of the commercial platforms are simply managed services with additional functionalities of the open-source platforms such as Confluent Platform, Amazon MSK, Aiven on top of Apache Kafka, Stream Native Cloud and Pandio on Apache Pulsar. Thus, only open-source platforms are considered in the thesis.

\section{Related Work}
%There are a number of articles and studies which compare and weigh different platforms and technologies. Many of them focus on evaluating the performance between platforms. There are comparisons of time and resource behavior of Apache Kafka and Apache Pulsar \cite{intorruk2019comparative} \cite{benchmarkkafkapulsar}, time efficiency between Apache Kafka and NATS Streaming \cite{benchmarknatskafka}.

There are a number of articles and studies which compare and weigh different platforms and technologies. However, most of them only focus on evaluating the performance between platforms \cite{benchmarkkafkapulsarrabbitmq} \cite{benchmarkfull} \cite{benchmarkkafkapulsar}.

Some other surveys cover more platforms and a wider range of evaluating aspects such as the comparison of Apache Kafka, Apache Pulsar and RabbitMQ from the company Confluent \cite{overallcomparekafka}. However, these assessments are conducted only briefly on the conceptual level. Apart from these studies, there is still lack of in-depth investigation into the differences of \acrshort{esp} platforms and their conformability with event-driven use cases and this is where the thesis will fill in.
 
\section{Contribution}
In this thesis, three open source \acrshort{esp} platforms, namely, Apache Kafka, Apache Pulsar and NATS Streaming are selected for evaluation based on preliminary measures and reasoning. Each platform is assessed against a set of criteria covering all important quality factors. The results are summarized in the form of a feature matrix with adjustable weighting factors of quality categories and features. Therefore, the matrix can be tailored to the needs of different users and adapted to individual use case to determine the most suitable platform for that case according to its priorities.

%In the evaluation, sample implementations and code snippets are presented to illustrate the features of the platforms. Moreover, best practices for each platform in different use cases are also drawn out. These can be used as a reference for actual implementation of applications on top of these platforms.
\section{Novatec Consulting GmbH}
This thesis is written with the support and supervision of Novatec Consulting GmbH. Novatec is an independent IT specialist with the headquarter located in Leinfelden-Echterdingen and 8 more branches across Germany and Spain. Its portfolio includes both development services and consultation in various aspects such as IT Architecture and Cloud, Data Intelligence, Business Process Management, Security.

The motivation for this thesis also arises from the real-world demand of the company to have a good and comprehensive evaluation of the most common ESP platforms for consultation purpose.   

\section{Organization of this Thesis}
The thesis is organized as follows. Chapter 2 gives a short theoretical background of the topics event-driven architecture, stream processing and \acrshort{esp} platforms. Chapter 3 enumerates prominent open-source \acrshort{esp} platforms currently available and furthermore derives criteria to choose the top three platforms which are considered in this thesis. Moreover, it also includes the elaboration of comparison metrics. After that, chapter 4 presents the evaluation of each platform against the comparison scheme and gives a discussion on the resulted feature matrix. Finally, the conclusion summarizes and proposes future improvement for the matrix.

